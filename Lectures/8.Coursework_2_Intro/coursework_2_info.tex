\documentclass{beamer}

\usetheme[]{Rochester}
\usecolortheme{beaver}
\usepackage[latin1]{inputenc}
\usepackage{graphics}

\author{Will Webberley}
\date{Autumn 2014}
\institute[COMSC]{Cardiff School of Computer Science and Informatics}



\title{Coursework Part 2}
\subtitle{CM2101: Human-Computer Interaction}

\begin{document}

\frame{\titlepage}

\frame{
    \frametitle{Overview}
    In this coursework you will essentially:
    \begin{itemize}
        \item Think up an \alert{application} for a system
        \item Choose a \alert{platform} to support the app
        \item Think up \alert{3} tasks that users of your app will do
        \item Create a medium (or high) fidelity interface \alert{prototype} to support the three tasks
        \item \alert{Evaluate} your prototype against Neilsen's heuristics
    \end{itemize}
}    

\frame{
    \frametitle{Submission}
    \textbf{This is an invidual coursework}
    \begin{itemize}
        \item Friday Week 7 (14th November @ 17:00) - \alert{Learning Central}
        \begin{itemize}
            \item Coversheet
            \item Supporting information
        \end{itemize}
        \item Monday Week 8 (your designated lab session) - \alert{Prototype demo}
        \begin{itemize}
            \item Demo your prototype to lab assessors
            \item Explain your heuristic evaluations
        \end{itemize}
    \end{itemize}
}

\frame{
    \frametitle{Mark breakdown}
    This coursework is worth 30\% of CM2101.\\
    Of the 30\%...
    \begin{itemize}
        \item \alert{55\%} - Prototype demonstration (in demo)
        \item \alert{35\%} - Heuristic evaluation explanation (in demo)
        \item \alert{10\%} - Supporting information (submitted to LC)
    \end{itemize}
    \vskip30pt
    See mark scheme for more complete breakdown.
}

\frame{
   \alert{Part 1: Prototype} 
}

\frame{
    \frametitle{Prototype - 1. App idea}
    \begin{itemize}
        \item Generate an idea for an app to design
        \item App must be able to support goals and task flows
        \item App must have a GUI
    \end{itemize}
}

\frame{
    \frametitle{Prorotype - 2. App platform}
    \begin{itemize}
        \item Consider the platform to support the app
    \end{itemize}
    \vskip10pt
    \textbf{Examples}
    \begin{itemize}
        \item Smartphone app (Android, iOS, Windows, ...)
        \item Tablet app (Android, iOS, Windows, ...)
        \item Desktop app (OS X, Windows, GNOME, ...)
        \item Web app
        \item Wearables app (Apple Watch, Android Wear, Pebble, ...)
    \end{itemize}
    \alert{e.g. \textit{A smartphone app for Android}}
}

\frame{
    \frametitle{Prototype - 3. App tasks}
    \begin{itemize}
        \item Identify \alert{3} tasks that users may perform
        \item Must have a flow
        \item Must not be \textit{too} trivial (try to incorporate a few distinct steps)
        \item Must not be too similar
   \end{itemize}
}

\frame{
    \frametitle{Prototype - 3. App tasks}
    \textbf{Examples}
    \begin{itemize}
        \item Cinema-booking tablet app for iOS (iPad)
        \begin{enumerate}
            \item Searching for upcoming showing times of movies
            \item Select a movie, choose ticket type(s), and select auditorium seat(s)
            \item Find an existing booking and modify the seat selection(s)
        \end{enumerate}
        \item Fitness-tracking smartphone app for Android
        \begin{enumerate}
            \item Adding a workout activity of a specific type, duration, and intensity to a fitness diary
            \item Searching for and adding other users of the app as friends
            \item Create workout `routines' by selecting from pre-existing workout activities
        \end{enumerate}
    \end{itemize}
}

\frame{
    \frametitle{Prototype - 4. Design implementation}
    \textbf{Wireframe or implement your design}
    \begin{itemize}
        \item Consider task analysis to help with task flow
        \item Design your app's interface to support the three tasks
    \end{itemize}
    \alert{Either...}
    \begin{itemize}
        \item Use a wireframing tool (e.g. Visual Paradigm)
        \item Wireframe screens required for completing the task
        \item Wireframe \textit{every} required component (including icons, dialog popups, menus, etc)
    \end{itemize}
    \alert{Or...}
    \begin{itemize}
        \item Implement an app
        \item More work, but some people are happier in code
        \item No further marks available
        \item But easier to demo/explain
    \end{itemize}
}

\frame{
    \frametitle{Prototype - 4. Design implementation}
    \begin{itemize}
        \item Recommended wireframing tool: Visual Paradigm
        \item Follow system-specific developer guidelines and be consistent
        \item Consider usability and design principles
        \item Consider the app's target audience
        \item Remember you will need to evaluate it afterwards
    \end{itemize}
}

\frame{
   \alert{Part 2: Heuristic Evaluation} 
}

\frame{
    \frametitle{Evaluation}
    \begin{itemize}
        \item Choose \alert{5} of Neilsen's heuristics to evaluate your design against
        \item When evaluating, consider the usability principle most relevant to the heuristic
        \item See coursework for examples
        \item See mark scheme to see how you will be assessed
        \item You do not need to write this up - just explain in demo
    \end{itemize}
}

\frame{
    \alert{Part 3: Supporting Information}
}

\frame{
    \frametitle{Supporting information}
    \textbf{The only written-up component}
    \begin{itemize}
        \item One-page overview of app (PDF):
        \begin{itemize}
            \item App description
            \item Audience, platform, etc.
            \item Description of your tasks
        \end{itemize}
        \item Visual `evidence' of your design (PDF):
        \begin{itemize}
            \item Visual Paradigm wireframe exports
            \item Or collection of screenshots documenting the design
        \end{itemize}
    \end{itemize}
}

\frame{
    \frametitle{In your demo...}
    \begin{itemize}
        \item 5 minutes (strict limit) to take assessor through your system
        \item Describe each task briefly and then demonstrate how it would be completed susing the design you've created
        \item As you go through, describe your evaluations
        \item Assessors will not prompt for this, so make sure you remember to talk about them
        \item Do not change your design between submission and demo 
    \end{itemize}
}

\end{document}
