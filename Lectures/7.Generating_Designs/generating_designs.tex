\documentclass{beamer}

\usetheme[]{Rochester}
\usecolortheme{beaver}
\usepackage[latin1]{inputenc}
\usepackage{graphics}

\author{Will Webberley}
\date{Autumn 2014}
\institute[COMSC]{Cardiff School of Computer Science and Informatics}



\title{Generating Designs and Prototyping}
\subtitle{CM2101: Human-Computer Interaction}

\begin{document}

\frame{\titlepage}
    
\frame{
    \frametitle{Generating designs}
    \begin{columns}
        \column{.5\textwidth}
            \begin{itemize}
                \item Iterative process important part of interaction design cycle
                \item `Entry point' for iterative cycle
                \item Requires either:
                \begin{itemize}
                    \item Initial needs / requirements
                    \item Evaluation from previous iteration
                \end{itemize}
                \item ... Before the design can be implemented
                \item Two types:
                \begin{itemize}
                    \item \alert{Conceptual} design
                    \item \alert{Phsycal} design
                \end{itemize}
            \end{itemize}
        \column{.5\textwidth}
            \includegraphics[width=5cm]{media/cycle_3.png}
    \end{columns}
}   

\frame{
    \frametitle{Conceptual design}
    \begin{center}
        \textit{A description of the proposed system in terms of ideas about what it should \alert{do}, \alert{behave}, and \alert{look like} in a form understandable by users.}
    \end{center}
    \begin{itemize}
        \item The overall idea of the system
        \item \alert{Task analysis} can be used in conceptual design
        \item What is it trying to solve and how to support \alert{requirements}
        \item Consider \textit{many} designs
        \begin{itemize}
            \item ``To get a good idea, get lots of ideas'' (Rettig, 1994)
        \end{itemize}
    \end{itemize}
}

\frame{
    \frametitle{Conceptual design: Perspectives}
    \textbf{1. Interaction mode}
    \begin{itemize}
        \item How does user invoke actions?
        \item e.g. conversing (dialogues), navigation, manipulating
    \end{itemize}
    \textbf{2. Interaction paradigm}
    \begin{itemize}
        \item How does the user interact with the system?       
        \item e.g. mobile devices, wearables, pervasive computing
    \end{itemize}
    \textbf{3. Interaction metaphor}
    \begin{itemize}
        \item How to make UI familiar \& understandable to users?
        \item e.g. telephone app 
    \end{itemize}
}

\frame{
    \frametitle{Conceptual design: Function and data}
    \textbf{Funtions}
    \begin{itemize}
        \item What functions will be exposed to the user?
        \item What can the system and used do?
    \end{itemize}
    \textbf{Relationships}
    \begin{itemize}
        \item How are the functions related?
        \item e.g. sequential or parallel
    \end{itemize}
    \textbf{Data}
    \begin{itemize}
        \item What input data is required?
    \end{itemize}
}

\frame{
    \frametitle{Physical design}
    \begin{center}
        \textit{The `physical' design and details of the system}
    \end{center}
    \begin{itemize}
        \item Screen considerations
        \item Device considerations
        \item Controls and data display
        \item Icons and text
        \item Style guides
    \end{itemize}
}


\frame{
    \frametitle{Prototype implementation}
    \begin{columns}
        \column{.5\textwidth}
            \begin{itemize}
                \item Implementation of design
                \item Fidelity of prototype increases with iterations
                \item Take ideas from design and implement into an interface (whether low-fi or high-fi)
                \item This interface can then be evaluated (heuristic, user, cognitive)
                \item ... Before a new design can be formed!
            \end{itemize}
        \column{.5\textwidth}
            % Update this image to `implementation' highlighted:
            \includegraphics[width=5cm]{media/cycle_3.png}
    \end{columns}
}

\frame{
    \frametitle{Prototypes}
    \textbf{What is a prototype generally?}
    \begin{columns}
        \column{.5\textwidth}
            \begin{itemize}
                \item A design \alert{hypothesis}
                \item Initial `versions' during development
                \item Communicating ideas
                \item Used for evaluation before improvement or release
                \item Often \textit{not} the released version
                \item e.g. a concept car
            \end{itemize}
        \column{.5\textwidth}
            \includegraphics[width=5cm]{media/i8_prototype.jpg}
    \end{columns}
}

\frame{
    \frametitle{Prototypes}
    \textbf{What is a prototype in interaction design?}
    \begin{columns}
        \column{.5\textwidth}
            \begin{itemize}
                \item Early versions of interfaces
                \item Storyboards
                \item Initial sketches of UI
                \item Video simulation
                \item Higher-fidelity mockups
                \item Early versions of implemented software
            \end{itemize}
        \column{.5\textwidth}
            \includegraphics[width=3cm]{media/vp_wireframe.png}
    \end{columns}
}

\frame{
    \frametitle{Why prototype?}
    \begin{itemize}
        \item \alert{Essential} for evaluation and gaining feedback
        \item Allows for testing ideas very quickly
        \item Answer questions and allow for choosing between designs
        \item Stakeholders can interact and understand prototypes more easily than written lists or drawings
        \item Allow team members to communicate ideas clearly
    \end{itemize}
}

\frame{
    \frametitle{What does prototyping capture?}
    \textbf{Prototyping encapsulates}
    \begin{itemize}
        \item Task design
        \begin{itemize}
            \item Methods
            \item Operator ordering (workflow)
            \item Goals and sub-goals
        \end{itemize}
        \item Technical issues (revealed when prototype is later \textit{evaluated}
        \item Usability issues (revealed when evaluated)
        \item Screen layouts
        \item Difficult-to-explain areas
    \end{itemize}
}

\frame{
    \frametitle{Prototype fidelity}
    \begin{center}
        \textit{How detailed is the prototype?}
    \end{center}
    \vskip20pt
    \alert{Lower fidelity}
    \begin{itemize}
        \item Less detailed, less realistic
        \item Good for: flow, functionality, terminology, screen content
        \item Bad for: Look \& feel, response, feedback
    \end{itemize}
    \vskip20pt
    \alert{Higher fidelity}
    \begin{itemize}
        \item More detailed, more realistic
        \item Good for: Look \& feel, feedback, colours/fonts
        \item Bad for: Quick re-design
    \end{itemize}
}

\frame{
    \frametitle{Prototyping: Low-fidelity}
    \begin{columns}
        \column{.7\textwidth}
            \begin{itemize}
                \item First-mid cycles in iterative design
                \item Very cheap - start with a sketch
                \item Very quick to develop and turn around (easily changed)
                \item Use a medium unlike final product (e.g. whiteboard, paper, post-its, digital sketches)
                \item Design team meet together and bash out ideas
                \item Limited functionality but...
                \begin{itemize}
                    \item Highlights ordering of tasks
                    \item Identifies where requirements met
                \end{itemize}
            \end{itemize}
        \column{.3\textwidth}
            \includegraphics[width=3cm]{media/lowfi_prototype.jpg}
    \end{columns}
}

\frame{
    \frametitle{Prototyping: Low fidelity}
    \textbf{How to evaluate and test?}
    \begin{itemize}
        \item When demoing
        \begin{itemize}
            \item Pointing = clicking
            \item Writing = typing
            \item Each sheet or card = new `screen'
            \item Can make small changes easily
        \end{itemize}
        \item Evaluation (user, heuristic, cognitive)
    \end{itemize}
}

\frame{
    \frametitle{Prototyping: Medium-fidelity}
    \begin{columns}
        \column{.7\textwidth}
            \begin{itemize}
                \item Mid-end cycles in iterative design
                \item More expensive and slower than low-fi
                \item But still allows for quicker implementation-evaluation cycle than high-fi
                \item 100\% digital (usually in a GUI builder or wireframing tool)
                \item Use UI components and controls specific to the platform
                \item Follow platform guidelines
                \item Can still incorporate new requirements
            \end{itemize}
        \column{.3\textwidth}
            \includegraphics[width=3cm]{media/vp_wireframe.png}
    \end{columns}
}

\frame{
    \frametitle{Prototyping: Medium-fidelity}
    \textbf{Done through \textit{wireframing}}
    \begin{itemize}
        \item Each wireframe represents an implemented screen design
        \item Sequence of wireframes indicates application/task flow
        \item Use GUI builders/wireframers
        \begin{itemize}
            \item Many allow system-specific components
        \end{itemize}
    \end{itemize}
}

\frame{
    \frametitle{Prototyping: High-fidelity}
    \begin{columns}
        \column{.5\textwidth}
            \begin{itemize}
                \item Final cycles in iterative design
                \item Usually as buggy implementations (or towards final product)
                \item Slower cycles
                \item More expensive (as a `working' product produced in each cycle)
                \item Inappropriate to incorporate new requirements (scope creep)
            \end{itemize}
        \column{.5\textwidth}
            \includegraphics[width=4cm]{media/swarm.png}
    \end{columns}
}

\frame{
    \frametitle{Prototype compromise}
    \textbf{Every prototype is inherently compromised}
    \begin{itemize}
        \item System dependent features not considered
        \begin{itemize}
            \item Response time
            \item Hardware functionality
            \item This is why evaluation is important afterwards
        \end{itemize}
        \item \alert{Horizontal} compromise (think: shallow HTA)
        \begin{itemize}
            \item Implement design for wide range of functions
            \item Each function is prototyped with limited detail
        \end{itemize}
        \item \alert{Vertical} compromise (think: deep HTA)
        \begin{itemize}
            \item Implement design for small number of functions
            \item Each function is prototyped with great detail
        \end{itemize}
        \item Compromises should be \alert{ignored}: products need to be \textit{engineered}
    \end{itemize}
}

\frame{
     \frametitle{Revision questions}
     \begin{enumerate}
        \item What is interaction design?
        \item What must happen before the `design' stage?
        \item Briefly describe the two parts to interaction design.
        \item What is a prototype with respect to interface design?        
        \item What does an interface prototype capture?
        \item Explain what is meant by a low-fidelity UI prototype and how it can be evaluated.
        \item What are the main advantages and disadvantages of high-fidelity prototyping in iterative design?
        \item Why are most prototypes compromised?
     \end{enumerate}
}

\frame{
    \frametitle{Summary}
    \begin{itemize}
        \item Interaction \alert{design}
        \begin{itemize}
            \item Conceptual design
            \item Physical design
        \end{itemize}
        \item Prototype \alert{implementation}
        \begin{itemize}
            \item Importance of prototypes and what they encapsulate
            \item How prototypes change further through iterative process
            \item Low- to high-fidelity prototyping
            \item Prototype compromise
        \end{itemize}
    \end{itemize}
}    

\end{document}
