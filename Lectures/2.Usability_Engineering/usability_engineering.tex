\documentclass{beamer}

\usetheme[]{Rochester}
\usecolortheme{beaver}
\usepackage[latin1]{inputenc}
\usepackage{graphics}

\author{Will Webberley}
\date{Autumn 2014}
\institute[COMSC]{Cardiff School of Computer Science and Informatics}



\title{User-Centric Interaction Engineering}
\subtitle{CM2101: Human-Computer Interaction}

\begin{document}

\frame{\titlepage}

\frame{
    \frametitle{What is an interface?}
    Generally, it is the plane of interaction between two media.
    \begin{description}[1980-2000]
        \item[1950-60] Interface at hardware only
        \item[1960-70] Interface at programming level
        \item[1970-80] Interface at terminal level
        \item[1980-2000] Interface at interaction dialogue (graphical displays)
        \item[2000-10] Interface becomes pervasive (RF tags, consumer electronics, embedded devices)
        \item[2010+] Interface replaced by \alert{experiences} (focus on tasks, emotions, elegance, social connections)
    \end{description}
}

\frame{
    \frametitle{}
    \centering{
        Therefore, we consider \alert{interaction design} instead of simply \alert{interface design}
    }
}

\frame{
    \frametitle{Users}
    The problem with users:
    \begin{itemize} 
        \item Systems typically have them 
        \item They're always `right'
        \item Users aren't the enemy
        \item Users aren't all the same
        \item Users can be prompted and instructed     
    \end{itemize}
}

\frame{
    \frametitle{User interaction}   
    Many interactions are small, but important:
    \begin{itemize}
        \item Automatic door
        \item Lift
        \item Travel ticket barrier
        \item Central heating
        \item Fridge
        \item Coffee machine
    \end{itemize}
    Designers need to consider the feedback loop to users.    
}

\frame{
    \frametitle{HCI design constraints}
    \centering
    \includegraphics[height=7cm]{media/constraints.png}   
}

\frame{
    \frametitle{Software development cycle}
    \begin{itemize}
        \item HCI is a key part of the software development cycle
        \item HCI also needs its own development cycle
        \item HCI aspects are drawn from the \alert{functional requirements} of a system
        \item If there is a task the user needs to do, then the system must allow it
        \item Tasks can be \alert{dynamically prioritised} through menus and options
    \end{itemize}
}

\frame{
    \frametitle{Waterfall: Traditional development cycle}
    \includegraphics[height=7cm]{media/waterfall.png}   
}

\frame{
    \frametitle{Waterfall: Traditional development cycle}  
    \begin{itemize}
        \item Each `iteration' becomes a release
        \item Effectively end-users become testers
        \item By this stage, it's too late
        \item If users don't like it, then they won't come back for future versions
    \end{itemize}
}

\frame{
    \frametitle{Waterfall: Traditional development cycle}
    Therefore, the waterfall is inappropriate for usability engineering, since:
    \begin{itemize}
        \item Design is likely to be `wrong' the first few times
        \item No user testing until the end
        \item Wasted effort on code written that needs to be thrown out
    \end{itemize}
    \vskip20pt
    Usability engineering needs to be much more agile:
    \begin{itemize}
        \item Iterative
        \item User-centric from the beginning
    \end{itemize}
}

\frame{
    \frametitle{Iterative and user-centric design}
    \includegraphics[height=4cm]{media/cycle.png}   
}

\frame{
    \frametitle{Iterative design}
    \begin{itemize}
        \item Use \alert{cheap} prototypes (whiteboards, paper, etc.) for \alert{early} iterations
        \item Use \alert{richer} prototypes later when \alert{changes become smaller}
        \item Ensure the `world' only sees mature iterations
        \item The more iterations: higher quality UX
    \end{itemize}   
}

\frame{
    \frametitle{User-centric design}
    \begin{itemize}
        \item Focus on the anticipated \alert{audience}
        \item Keep in mind the \alert{goals} and \alert{tasks} the users will need to do
        \item \alert{Consult} users in \alert{each} iteration
        \item Involve non-developmental users as \alert{evaluators} in \alert{each} iteration
        \item Keep in mind \alert{human factors} and the abilities of the users
    \end{itemize}
}

\frame{
    \frametitle{LUCID framework}
    \alert{Logical User-Centric Interaction Design}
    \textit{- Charlie Kreizberg, Cognetics Corporation}
}

% LUCID stuff

\frame{
    \frametitle{User evaluation}
    \alert{User evaluation} is the first interaction evaluation studied on this course.
    This will be useful for your coursework, but there are also other techniques we'll cover later.
    \vskip10pt
    \centering{
        \includegraphics[height=4cm]{media/cycle_2.png}
    }
}

\frame{
    \frametitle{User evaluation: `think aloud'}
    Generally: a user speaks their thoughts whilst completing a task.
    \begin{block}{Think aloud recipe}
        \begin{enumerate}
            \item Tester selects a task to be completed for a given system
            \item Tester is told what they are required to do
            \item User is given access to the system's interface
            \item User navigates through the system and tries to complete the task
            \item User speaks \alert{all} thoughts aloud
            \item Tester records all spoken thoughts for later evaluation 
        \end{enumerate}
    \end{block}
}

\frame{
    \frametitle{User evaluation: `think aloud'}
    \begin{itemize}
        \item User should try and speak aloud all thoughts
        \begin{itemize}
            \item Frustrations (``Ugh... Why did \textit{that} happen?'')
            \item What they think is happening
            \item What they're trying to do
            \item Messages encountered (and perception of these)
            \item Why a particular action was taken (``I tapped here because...'')
            \item Things that are confusing (``What's the difference between these two options?'')
            \item Questions that arise in the mind
        \end{itemize}
        \item Tester should stay quiet and shouldn't respond to questions
        \item If user is quiet, tester should prompt (``Why did you tap that?'', ``What are you thinking?'')
    \end{itemize}
}

\frame{
    \frametitle{`Think aloud' method}
    \begin{columns}
        \column{.5\textwidth}
            \begin{block}{Pros}
                \begin{itemize}
                    \item Gives insights into thoughts
                    \item Usable in early iterations
                    \item Understand what the interface \textit{says} to the user
                    \item Understand what feels natural to the user and what doesn't
                    \item Highlights areas of frustration or confusion
                    \item Widely-used evaluation method
                \end{itemize}
            \end{block}
        \column{.5\textwidth}
            \begin{block}{Cons}
                \begin{itemize}
                    \item May alter the way users approach the task
                    \item It feels unnatural
                    \item Hard to talk if concentrating
                    \item Tasks take longer (slows by about 17\%)
                \end{itemize}
            \end{block}
    \end{columns}
}

\frame{
    \frametitle{`Think aloud' setup}
    \begin{itemize}
        \item User should have the procedure and test explained to them
        \item User should be relaxed
            \begin{itemize}
                \item Demo `think aloud' using unrelated task (e.g. looking up train times)
                \item User should practice using an unrelated task
            \end{itemize}
        \item Ensure quiet, comforable room. Turn phones to silent and prevent anticiapted interruptions.
        \item Test may be recorded visually 
    \end{itemize}
}

\frame{
    \frametitle{After a `think aloud' test}
    \begin{itemize}
        \item Use notes to evaluate performance of interactions
        \item Where were there positive thoughts? (``This makes sense'', ``That's what I thought would happen'')
        \item Where were there negative thoughts? (``How am I supposed to be able to do this?'')
        \item Use notes to help improve next iteration
    \end{itemize}
}

\end{document}
